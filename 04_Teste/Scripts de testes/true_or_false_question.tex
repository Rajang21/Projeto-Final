\documentclass[12pt,varwidth=16cm,border=1pt]{standalone}

\usepackage{listings}
\renewcommand{\lstlistingname}{Listagem}
\lstset{
  language=python,
  frame=single,
  breaklines=true,
  inputencoding=utf8,
  extendedchars=true,
  literate={á}{{\'a}}1
           {à}{{\`a}}1
           {ã}{{\~{a}}}1
           {õ}{{\~{o}}}1
           {é}{{\'e}}1
           {í}{{\'i}}1
           {ó}{{\'o}}1
           {ú}{{\'u}}1
           {ç}{{\c{c}}}1
           {ê}{{\^{e}}}1
           {º}{{\textsuperscript{o}}}1,
  showstringspaces=false,
  basicstyle=\scriptsize\ttfamily,
  columns=flexible,
  keepspaces=true,
  %keywordstyle=\scriptsize\ttfamily\color{idlebuiltins},
  %keywords=[2]{type, ord, chr, bin, int},
  %keywordstyle={[2]\scriptsize\ttfamily\color{idlebuiltins}},
  commentstyle=\scriptsize\ttfamily
}

\renewcommand{\labelenumi}{\alph{enumi})}

\begin{document}

% este documento é um template.

% este documento é usado pelo script make_random_versions.py para
% criar as versões desta pergunta

Considere a classe \verb+CorRGB+, implementada durante as aulas, no módulo
\verb+cor_rgb_xxxxx.py+ (onde \verb+xxxxx+ representa o seu número de
aluno), e cuja especificação está disponível no moodle.

Considere também o programa Python 3 que se segue, na mesma
pasta/diretoria do módulo \verb+cor_rgb_xxxxx.py+ (de forma a que o
\verb+import+ seja executado sem erros). Ignore a variável \verb+seed+
e a função \verb+random_float+, que se destinam
exclusivamente à geração de números pseudo-aleatórios.

\lstinputlisting{program.py}

Acrescente a este programa:

\begin{enumerate}

\item A lista \verb+s4+. O elemento da lista \verb+s4+, em cada
  índice, é um objeto \verb+CorRGB+ que resulta da soma dos objetos
  \verb+CorRGB+ das listas \verb+s1+ e \verb+s2+, no mesmo
  índice.

\item A lista \verb+s5+. O elemento da lista \verb+s5+, em cada
  índice, é um objeto \verb+CorRGB+ que resulta do produto dos objetos
  \verb+CorRGB+ das listas \verb+s1+ e \verb+s2+, no mesmo
  índice.

\item A lista \verb+s6+. O elemento da lista \verb+s6+, em cada
  índice, é um objeto \verb+CorRGB+ que resulta do produto do objeto
  \verb+CorRGB+ na lista \verb+s1+ pelo \verb+float+ da
  \verb+s3+, no mesmo índice.

\end{enumerate}

Indique se é verdadeiro ou falso.

\end{document}




